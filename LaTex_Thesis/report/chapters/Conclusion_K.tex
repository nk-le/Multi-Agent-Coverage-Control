\chapter{Conclusion and Future Work}
%This thesis studied the problem of coverage control of a group of underactuated Wheeled Mobile Robots (WMR) under the state and input constraint. By applying a nonlinear scaling factor, the control law can handle input saturation to ensure the stability of the coverage problem. Besides, the thesis also proposed a switching condition, based on the theorem of barrier Lyapunov function, to guarantee that the state constraint is never violated. The method was proven and simulated under many scenarios with varying parameters and complexity. From a practical point of view, this control law is decentralized, applicable for any hardware specification of WMR, and for any convex coverage region. This corresponds to the motivation of the thesis that a control law can find a compromise to deal with all constraints at the same and ensure the operational performance. \\
%During the project, we note some challenges that open the potential research directions as future work. The first one refers directly to the proposed control law. By means of proposing a switching control, we do not consider the adjacent agents. Indeed, the movement of the neighbor agents is what makes the guarantee of the state feasibility challenging. Since the barrier Lyapunov function also depends on the position of these agents, we were not able to analyze the function analytically. This motivates us to find strategies that consider the uncertainties due to the neighbor agents to ensure all sufficient conditions of the proposition 2.\\
%One of the most important future work is conducting experiments to assess the control method. Since all the constraints considered in this project are strongly related to real situations, we are looking forward to evaluating the performance and the reliability from the practical aspect.\\
%Furthermore, the proposed controller has a limitation that it is only applicable to cover a convex region. Therefore, we are inspired to find a controller for a non-convex coverage that can also ensure all of the constraints. 

This thesis studies the problem of coverage control executed by a group of Wheeled Mobile Robots (WMR) under the state and input constraint. By applying a non-linear scaling factor, the control law can handle the input saturation to ensure the stability of the coverage problem. Besides, the thesis also proposes a switching condition, based on the theorem of barrier Lyapunov function, to guarantee that the state constraints are never violated. The method is proven and simulated under many scenarios with varying parameters and complexity. From a practical point of view, this control law is decentralized, applicable for any hardware specification of WMR,
and for any convex region. This corresponds to the motivation of the thesis that a control law can find a compromise to deal with all constraints at the same time and ensure the operational performance. \\
During the project, we note some challenges that open the potential research directions as future work. The first one refers directly to the proposed control law. By
means of applying a switching condition, we do not consider the adjacent agents. Indeed, the movement of the neighbor agents is what makes the guarantee of the state feasibility challenging. Since the barrier Lyapunov function also depends on the position of these agents, we were not able to analyze the function analytically. This motivates us to find strategies that consider the uncertainties due to the neighbor agents to ensure all sufficient conditions of the proposition 2. \\
One of the most important future work is conducting experiments to assess the control method. Since all the constraints considered in this project are strongly
related to real situations, we are looking forward to evaluating its performance and reliability from the practical aspect. \\
Furthermore, the proposed controller has a limitation that it is only applicable for convex regions. Therefore, we are inspired to find a control method for a non-convex coverage control that can ensure all of the constraints. \\
